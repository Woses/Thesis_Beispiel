\chapter{Methodik und Stand der Forschung}

\section{Ziel und Studiendesign}
Ziel dieses Kapitels ist es, die Vorgehensweise der Literaturauswertung transparent zu machen und den aktuellen Stand der Forschung strukturiert zu synthetisieren. Der Fokus liegt auf Mechanismen der Farbentstehung bei Gräsern, insbesondere Pigmentabsorption, Blattoptik, Photoprotektion und Wahrnehmungsaspekte. Die Arbeit folgt einer narrativen, qualitativ-analytischen Synthese auf Basis gezielter Literaturrecherche.

\section{Literaturrecherche}
Die Recherche erfolgte iterativ in einschlägigen Datenbanken und Suchmaschinen. Die Kombination aus kontrollierten Begriffen und freien Suchphrasen zielte auf Übersichtsarbeiten und zentrale Primärstudien.

\subsection{Datenbanken und Suchhorizont}
\begin{tblr}{
  colspec={lX},
  row{1} = {font=\bfseries},
  rows = {abovesep=2pt, belowsep=2pt}
}
Quelle / Datenbank & Rahmen \\
Google Scholar      & Breite Abdeckung, Vorselektion über Titel/Abstract. \\
Web of Science      & Fachlich kuratierte Journals, Zitationsnetzwerke. \\
Institutionelle Repositorien & Zugriff auf historische Primärquellen und Lehrmaterial. \\
\end{tblr}

\subsection{Suchstrings (Beispiele)}
\begin{itemize}
  \item \textit{“chlorophyll absorption”} \textit{AND} \textit{“green gap”} \textit{AND} \textit{“leaf optics”}
  \item \textit{“photosystem electron transport”} \textit{AND} \textit{“ATP NADPH”}
  \item \textit{“green light canopy penetration”} \textit{AND} \textit{“shade tolerance”}
  \item \textit{“carotenoids photoprotection”} \textit{AND} \textit{“nonphotochemical quenching”}
\end{itemize}

\subsection{Ein- und Ausschlusskriterien}
\begin{tblr}{
  colspec={lX},
  row{1} = {font=\bfseries},
  rows = {abovesep=2pt, belowsep=2pt}
}
Einschluss & Peer-reviewte Artikel, Kapitel oder Standardlehrwerke mit direktem Bezug zu Pigmentabsorption, Blattoptik, Photosynthesekinetik oder Wahrnehmungsaspekten. \\
Ausschluss & Rein technische Anwendungsberichte ohne Bezug zu Mechanismen; Studien ohne nachvollziehbare Methodik; Duplikate. \\
\end{tblr}

\subsection{Screening und Auswahl}
Zunächst wurden Titel und Abstract gescreent (Relevanz, Qualität). Danach Volltexte geprüft und bibliographische Rückwärtssuche genutzt. Zentralbelege wurden priorisiert, ergänzend wurden Übersichtsarbeiten herangezogen \parencite{meyer2018photosynthese, schmidt2015chlorophyll, gao2010lightabsorption, zhao2012chlorophyll, renoult2017evolution}.

\section{Datenextraktion und Auswertung}
Für jedes Werk wurden Kernaussagen zu (i) Pigmenten und Spektren, (ii) Blattoptik/Gewebe, (iii) Photoprotektion und (iv) Wahrnehmung erfasst. Widersprüche wurden tabellarisch festgehalten und in der Diskussion adressiert. Die Synthese folgt einem mechanistischen Raster (vom Photon zum visuellen Eindruck) und verknüpft Befunde über Skalen (Blatt, Bestand, Wahrnehmung).

\section{Bias und Limitationen}
Die narrative Synthese ist anfällig für Auswahl- und Publikationsbias. Dem wurde begegnet durch (i) Abdeckung mehrerer Datenbanken, (ii) Rückwärtssuche, (iii) Bevorzugung etablierter Lehr- und Übersichtsquellen sowie (iv) explizite Benennung kontroverser Punkte. Nicht alle Spezialfälle (etwa extremstandortangepasste Arten) werden abgedeckt.

\section{Stand der Forschung}
\subsection{Konsenslagen}
\begin{itemize}
  \item \textbf{Pigmentabsorption}: Chlorophyll absorbiert stark im blauen und roten Bereich, schwach im grünen Bereich; akzessorische Pigmente erweitern das Spektrum und schützen vor Überanregung \parencite{meyer2018photosynthese, schmidt2015chlorophyll, gao2010lightabsorption}.
  \item \textbf{Photosynthesekette}: Wasseroxidation an PS\,II, Elektronentransport, Reduktion zu NADPH an PS\,I und ATP-Bildung sind etabliert \parencite{zhao2012chlorophyll}.
  \item \textbf{Wahrnehmung}: Das Maximum der menschlichen Zapfensensitivität liegt im Grünbereich; das unterstützt die saliente Wahrnehmung von Vegetation \parencite{renoult2017evolution}.
\end{itemize}

\subsection{Kontroversen und Debatten}
\begin{itemize}
  \item \textbf{Rolle von grünem Licht}: Am Einzelblatt geringer absorbiert, im Bestand jedoch wichtige Tiefeindringung und potenzieller Beitrag zur Produktivität in unteren Blattlagen \parencite{zhao2012chlorophyll}.
  \item \textbf{Photoprotektion vs. Effizienz}: Trade-offs zwischen maximaler Ausnutzung roter/blauer Photonen und Wärmeabfuhr bzw. nichtphotochemischer Löschung \parencite{gao2010lightabsorption}.
  \item \textbf{Wahrnehmungsökologie}: Kausalrichtung zwischen dominanter Umweltfarbe und visueller Sensitivität bleibt Gegenstand der Diskussion \parencite{renoult2017evolution}.
\end{itemize}

\subsection{Forschungslücken}
Benötigt werden vergleichende Studien, die (i) spektrale Profile über Blattlagen in Beständen quantifizieren, (ii) Photoprotektionsdynamik unter variierenden Lichtfluktuationen erfassen und (iii) Wahrnehmungseffekte artübergreifend einordnen.

\section{Verknüpfung zur Struktur der Arbeit}
Die in der Einleitung skizzierte Motivation wird hier operationalisiert: Kapitel~\emph{Theoretischer Hintergrund} liefert die mechanistische Basis; Kapitel~\emph{Anwendungen und Fallbeispiele} demonstriert Mess- und Auswertungswege; Kapitel~\emph{Diskussion} ordnet kontroverse Befunde und Grenzen ein.
