\usepackage{setspace}
\KOMAoptions{ twoside=true, open=any, headings=optiontoheadandtoc,parskip=true,fontsize=12pt,parskip=half-}
\usepackage{geometry}
\geometry{
	paperwidth=210mm,
	paperheight=297mm,
	layout=a4paper,
	lmargin=23mm, % margin auf der rechten Seite
	rmargin=25mm, % margin auf der linken Seite
	tmargin=18mm,
	textheight=250mm,
	textwidth=154mm,
	includefoot,
	footskip=30pt,
	layoutoffset=3mm
}




%%%%%% Print %%%%%%%%%%%%%%%%%%%%%%%%%%%%%%%%%%%%%%%%%%%%%%%%%%%%%%%%

% Lege die ICC-Datei PSOcoated_v3.icc ins Projektverzeichnis
%\usepackage[x-4]{pdfx} % erzeugt PDF/X-4 mit OutputIntent
% pdfx nimmt standardmäßig die im Projekt liegende ICC als OutputIntent, z. B. PSOcoated_v3.icc

%%%%%% Portraits %%%%%%%%%%%%%%%%%%%%%%%%%%%%%%%%%%%%%%%%%%%%%%%%%%%%

\usepackage{float}
\usepackage{wrapstuff}
\usepackage{wrapfig}


%%%%%% Section titles %%%%%%%%%%%%%%%%%%%%%%%%%%%%%%%%%%%%%%%%%%%%%%%

\setcounter{secnumdepth}{0}
\setcounter{tocdepth}{4}

\RedeclareSectionCommand[
	afterskip = {0.4ex plus 0.2ex},
	beforeskip = {-1ex plus -1ex minus -.2ex},
	font={\LARGE},
]{section}

\RedeclareSectionCommand[
	afterskip = {.4ex plus .2ex},
	beforeskip = {-0.25ex plus -1ex minus -.2ex},
	font={\Large},
]{subsection}
	
\DeclareNewSectionCommand[
	afterskip = {.4ex plus .2ex},
	beforeskip = {-0.25ex plus -1ex minus -.2ex},
	indent=0pt,
	font={\LARGE},
	style=section,
	level=1,
	tocindent=0pt,
	toclevel=1,
	tocnumwidth=2em,
	tocstyle=chapter,
]{sectionPart}

\RedeclareSectionCommand[
afterskip = {-1em},
beforeskip = {0ex plus -1ex minus .2ex},
runin=false,
]{paragraph}

\setparsizes{0pt}%infdent
{9pt}%vskip
{0pt plus 1fil}


%%%%%% Section headings %%%%%%%%%%%%%%%%%%%%%%%%%%%%%%%%%%%%%%%%%%%%

\usepackage{ifthen}
\newcounter{secautolabel}
\AddtoDoHook{heading/endgroup}{\setautolabel}
\newcommand*{\setautolabel}[1]{%
  	\stepcounter{secautolabel}%
  	\label{sec:autolabel:\thesecautolabel}%
	\def\sname{#1}
	\ifthenelse{\equal{#1}{sectionPart}}{
		\def\sname{section}
	}{}
  	\expandafter\xdef\csname \sname title\endcsname{%
    	\noexpand\nameref{sec:autolabel:\thesecautolabel}%
  	}%
}


%%%%%% Font, Encoding etc. %%%%%%%%%%%%%%%%%%%%%%%%%%%%%%%%%%%%%%%%%%

\usepackage[utf8]{inputenc}
\usepackage[T1]{fontenc}
\usepackage[ngerman]{babel}
\usepackage{amsmath,amsfonts,amssymb}
\usepackage[default,light,lf]{FiraSans}
\usepackage{pifont}


%%%%%% Schritsatz %%%%%%%%%%%%%%%%%%%%%%%%%%%%%%%%%%%%%%%%%%%%%%%%%%%%

% Blocksatz aktiv + bessere Trennung
\usepackage{microtype}        % Mikrotypografie für schönere Zeilen
\emergencystretch=2em         % Puffer gegen overfull lines

% Flattersatz & Trennung aus
%\usepackage{ragged2e}
%\AtBeginDocument{\RaggedRight}      % global linksbündig (kein Blocksatz)
%\usepackage[none]{hyphenat}         % Silbentrennung aus
%\emergencystretch=1em               % etwas Puffer gegen overfull lines
%\setlength{\RaggedRightParindent}{0pt}


%%%%%% Colors %%%%%%%%%%%%%%%%%%%%%%%%%%%%%%%%%%%%%%%%%%%%%%%%%%%%%%%

\usepackage{xcolor,colortbl}

\definecolor{fscolor}{RGB}{11,161,226}
\definecolor{petrol}{HTML}{216477}
\definecolor{lecegreen}{HTML}{0aa3e4}


%%%%%% tikz %%%%%%%%%%%%%%%%%%%%%%%%%%%%%%%%%%%%%%%%%%%%%%%%%%%%%%%%%

\usepackage{tikz}
\usetikzlibrary{calc}
\usetikzlibrary{shapes.geometric}
\usetikzlibrary{decorations}


%%%%%% Tabellen %%%%%%%%%%%%%%%%%%%%%%%%%%%%%%%%%%%%%%%%%%%%%%%%%%%%%

\usepackage{tabularx}
\usepackage{makecell}
\usepackage{tabularray}
\usepackage{multicol}
%\usepackage{tocbasic}        % Inhaltsverzeichnis verändern schiebt es nach unten oder oben

%%%%%% Sonstiges %%%%%%%%%%%%%%%%%%%%%%%%%%%%%%%%%%%%%%%%%%%%%%%%%%%%

\usepackage{qrcode}

\usepackage[]{graphicx}

\usepackage[export]{adjustbox}% http://ctan.org/pkg/adjustbox

\usepackage{caption}
\usepackage{subcaption}

\usepackage{array}
\usepackage{forarray}
\usepackage{forloop}

\usepackage{enumitem}
\usepackage{etoolbox}
\usepackage[german=quotes]{csquotes}
\usepackage{scrlayer-fancyhdr}

\usepackage{microtype}
\usepackage{lipsum}           % Platzhaltertext
\usepackage{pdfpages}         % Für PDF-Einbindung
\usepackage{setspace}



% URLs besser umbrechen (Option muss vor hyperref rein)
\PassOptionsToPackage{hyphens}{url}


\usepackage[
	linktoc=black,
	linkcolor=.,
	urlcolor=fscolor,
	breaklinks=true,
	colorlinks,
	bookmarks,
	plainpages=false,
	citecolor=gray,
]{hyperref}
\urlstyle{same}
\usepackage{cleveref}

% URL-Umbrüche noch flexibler
\Urlmuskip=0mu plus 2mu


% Checkliste
\newlist{todolist}{itemize}{2}
\setlist[todolist]{label=$\square$}


\newcounter{pagehead}


%%%%%%%%%%%%%%%%%% Citations %%%%%%%%%%%%%%%%%%%%%%%%%%%%%%%%%%%%%%%%
\usepackage[backend=biber,style=authoryear,maxcitenames=2,maxbibnames=99]{biblatex}
\addbibresource{literatur.bib}
